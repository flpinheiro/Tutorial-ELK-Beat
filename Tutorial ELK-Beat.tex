%% Based on a TeXnicCenter-Template by Gyorgy SZEIDL.
%%%%%%%%%%%%%%%%%%%%%%%%%%%%%%%%%%%%%%%%%%%%%%%%%%%%%%%%%%%%%

%------------------------------------------------------------
%
\documentclass{article}%
%Options -- Point size:  10pt (default), 11pt, 12pt
%        -- Paper size:  letterpaper (default), a4paper, a5paper, b5paper
%                        legalpaper, executivepaper
%        -- Orientation  (portrait is the default)
%                        landscape
%        -- Print size:  oneside (default), twoside
%        -- Quality      final(default), draft
%        -- Title page   notitlepage, titlepage(default)
%        -- Columns      onecolumn(default), twocolumn
%        -- Equation numbering (equation numbers on the right is the default)
%                        leqno
%        -- Displayed equations (centered is the default)
%                        fleqn (equations start at the same distance from the right side)
%        -- Open bibliography style (closed is the default)
%                        openbib
% For instance the command
%           \documentclass[a4paper,12pt,leqno]{article}
% ensures that the paper size is a4, the fonts are typeset at the size 12p
% and the equation numbers are on the left side
%
\usepackage{amsmath}%
\usepackage{amsfonts}%
\usepackage{amssymb}%
\usepackage{graphicx}
%-------------------------------------------
% pacotes do portugues
\usepackage[portuguese]{babel}% fazemos com que o compilador traduza express�es como �table of contents�, �chapter� ou �appendix� para o portugu�s e passa tamb�m a escrever as datas com os nomes dos meses em portugu�s.
\usepackage[latin1]{inputenc}% permitimo-nos o uso de caracteres com acentos e cedilhas.
%-------------------------------------------
\usepackage{hyperref} % 
\hypersetup{
    colorlinks=true,
    linkcolor=blue,
    filecolor=magenta,      
    urlcolor=cyan,
}
\urlstyle{same}
%-------------------------------------------
\newtheorem{theorem}{Theorem}
\newtheorem{acknowledgement}[theorem]{Acknowledgement}
\newtheorem{algorithm}[theorem]{Algorithm}
\newtheorem{axiom}[theorem]{Axiom}
\newtheorem{case}[theorem]{Case}
\newtheorem{claim}[theorem]{Claim}
\newtheorem{conclusion}[theorem]{Conclusion}
\newtheorem{condition}[theorem]{Condition}
\newtheorem{conjecture}[theorem]{Conjecture}
\newtheorem{corollary}[theorem]{Corollary}
\newtheorem{criterion}[theorem]{Criterion}
\newtheorem{definition}[theorem]{Definition}
\newtheorem{example}[theorem]{Example}
\newtheorem{exercise}[theorem]{Exercise}
\newtheorem{lemma}[theorem]{Lemma}
\newtheorem{notation}[theorem]{Notation}
\newtheorem{problem}[theorem]{Problem}
\newtheorem{proposition}[theorem]{Proposition}
\newtheorem{remark}[theorem]{Remark}
\newtheorem{solution}[theorem]{Solution}
\newtheorem{summary}[theorem]{Summary}
\newenvironment{proof}[1][Proof]{\textbf{#1.} }{\ \rule{0.5em}{0.5em}}

\begin{document}

\title{Tutorial ELK-Beat}
\author{Felipe Lu�s Pinheiro \and Marcelo Antonio}
\date{\today}
\maketitle

\begin{abstract}
Este tutorial foi desenvolvido para ajudar o desenvolvedores do Compuletra a implementar aplica��es em microservi�os o sistema da Elastic, ELK Stack junto com o Filebeat, para realizar o log do sistema. 

Usamos como exemplo uma aplica��o .Net com o Serilog, por�m esse sistema pode ser usado por qualquer tipo de aplica��o com qualquer linguagem de programa��o. 
\end{abstract}

\section{Introdu��o}

Nesta se��o faremos uma r�pida introdu��o das ferramentas e tecnologias utilizadas para o desenvolvimento desse projeto.

\subsection{ELK Stack + Filebeat}

O ELK Stack e o filebeat s�o mantidos e desenvolvidos pela \href{https://www.elastic.co/}{Elastic}\footnote{\url{http://www.sharelatex.com}} sendo que a vers�o mais recente � a 7.6 e � composto de:

- \href{https://www.elastic.co/guide/en/elasticsearch/reference/current/index.html}{Elasticsearch}\footnote{\url{https://www.elastic.co/guide/en/elasticsearch/reference/current/index.html}}: um banco de dados nosql, que armazena o dados de forma indexada em formato sql de modo a ser r�pido e leve e que possui uma api RESTfull.

- \href{https://www.elastic.co/guide/en/logstash/current/index.html}{Logstash}\footnote{\url{https://www.elastic.co/guide/en/logstash/current/index.html}}: um ingestor de dados que pode receber dados de v�rias fontes diferentes trata-los e reencaminha-los para od destinos apropriados com um formato mais adequado para o destino.

- \href{https://www.elastic.co/guide/en/kibana/current/index.html}{Kibana}\footnote{\url{https://www.elastic.co/guide/en/kibana/current/index.html}} � um motor gr�fico que serve para gerenciar os dados do Elasticsearch de forma f�cil com o uso da api RESTfull. 

- \href{https://www.elastic.co/guide/en/beats/filebeat/current/filebeat-overview.html}{Filebeat}\footnote{\url{https://www.elastic.co/guide/en/beats/filebeat/current/filebeat-overview.html}}: � um despachador de logs, desempenha seu papel ao lado da aplica��o onde l� os logs gerado pela aplica��o e despacha para o destino apropriado de modo automatico e autonomo, existem outras vers�es de beat da elastic, tal como o metricbeat que serve para despachar metricas de uso e o Auditbeat que serve para auditar a atividade dos usu�rios e dos processos, para mais informa��es acesse a \href{https://www.elastic.co/guide/en/beats/libbeat/7.6/index.html}{Beats plataform}\footnote{\url{https://www.elastic.co/guide/en/beats/libbeat/7.6/index.html}}. 

\subsection{.Net Framework}

.Net Framework � um framewor

\subsection{Serilog}

\subsection{Docker}

\end{document}
